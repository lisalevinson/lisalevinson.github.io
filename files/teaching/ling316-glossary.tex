% For slides
\documentclass[12pt,letterpaper]{article}
%\usetheme{Derwent}
%\setbeamercovered{invisible}

% For handout
%\documentclass[handout]{beamer}
%\usetheme{handout}

\usepackage{natbib}
\bibpunct{(}{)}{,}{a}{}{,}
%\newcommand\newblock{\hskip .11em \@plus.33em \@minus.07em}%

\usepackage[latin1]{inputenc}
\usepackage{times}
\usepackage[T1]{fontenc}
\usepackage{xypic}
\usepackage{stmaryrd}
\usepackage{qtree}

\usepackage{graphicx}
\usepackage{linguex}
\usepackage{gb4e}

%\renewcommand{\refdash}{}

\newcommand{\ve}[0]{\ensuremath{\;\&\;}}
\newcommand{\set}[1]{\{#1\}}
\newcommand{\oset}[1]{\langle #1\rangle}
\newcommand{\card}[1]{\ensuremath{|#1|}}
\newcommand{\sem}[1]{\ensuremath{\llbracket #1 \rrbracket}}
\newcommand{\pred}[1]{\textsc{#1}}

\begin{document}
\begin{center}
\begin{Large}Glossary and Small Notes\end{Large}\\
\vspace{.2in}
\begin{large}LING 316 Aspects of Meaning with Professor Lisa Levinson\end{large}\\
\vspace{.2in}
\today
\end{center}

This document will grow throughout the semester, as we cover different topics. Definitions and notes may be revised as we progress as well! The date that the document was generated will always be updated at the top.

\section{Foundations}

\begin{itemize}
  \item Model theory -
  \item Principle of Compositionality - \citep{frege1892}
  \item    
\end{itemize}


%\input{foundations}

%\begin{exe}
%\ex[\label{blah}]{blah}
%\end{exe}

%in example \ref{blah}

%OOO and i can even pull up references from the input file: %\ref{bli}

%can i keep writing?

\setlength{\bibsep}{0ex}
\bibliographystyle{linquiry2}
\bibliography{C:/Users/lisalev/GoogleDrive/lbl-zotero}

\end{document}
